%%%%%%%%%%%%%%%%%%%%%%%%%%%%%%%%%%%%%%%%%
% Original author:
% Linux and Unix Users Group at Virginia Tech Wiki
% (https://vtluug.org/wiki/Example_LaTeX_chem_lab_report)
% Modified by: Hector F. Jimenez S, for the Digital Electronics Laboratory.
% License:
% CC BY-NC-SA 3.0 
%%%%%%%%%%%%%%%%%%%%%%%%%%%%%%%%%%%%%%%%%
%----------------------------------------
%	PACKAGES AND DOCUMENT CONFIGURATIONS
%---------------------------------------
\documentclass[paper=a4, fontsize=12pt]{article} 		% A4 paper and 11pt font size
\usepackage[T1]{fontenc} 								% Use 8-bit encoding that has 256 glyphs
%\usepackage{fourier}		 							% Use the Adobe Utopia font for the document 
\usepackage[spanish,english]{babel}						% Spanish Language, templates uses some sections in english.
\selectlanguage{spanish}								% main language.
\PassOptionsToPackage{spanish}{babel}
%\renewcommand{\figurename}{Figura}						% Force rename of figure.
%\renewcommand{\figurename}{Fig.}
\usepackage[figurename=Fig.]{caption}
\usepackage[utf8]{inputenc}								% tildes for spanish language.
\usepackage{amsmath,amsfonts,amsthm} 					% Math packages.
\usepackage{minted}										% For syntax highlighting.
\usepackage{float}										% Image will be in the same place as you want.!!! x-/
\usepackage{sectsty} 									% Allows customizing section commands
\allsectionsfont{\centering \normalfont\scshape}	   	% Make all sections centered, the default font and small caps
\usepackage{hyperref}
\hypersetup{											%Setups the false color and borders.
    colorlinks=false,
    pdfborder={0 0 0},
}
\newcommand\fnurl[2]{%									% set a simple and quick footnote command and include url.
\href{#2}{#1}\footnote{\url{#2}}%	
}
\usepackage{graphicx}									% Import easyly images.
\graphicspath{ {./images/} }							% Where to look for the images.
\DeclareGraphicsExtensions{.pdf,.png,.jpg}				% Graphics Extension to be used
\usepackage[notes,backend=biber]{biblatex-chicago}		% Bibliography and references.
\bibliography{biblio}									% bibliography filename.
\usepackage{fancyhdr} 									% Custom headers and footers
\pagestyle{fancyplain} 									% Makes all pages in the document conform to the custom headers and footers
\fancyhead{} 											% No page header
\fancyfoot[L]{} 										% Empty left footer
\fancyfoot[C]{} 										% Empty center footer
\fancyfoot[R]{\thepage} 								% Page numbering for right footer
\renewcommand{\headrulewidth}{0pt} 						% Remove header underlines
\renewcommand{\footrulewidth}{0pt} 						% Remove footer underlines
\setlength{\headheight}{13.6pt} 					    % Customize the height of the header
\numberwithin{equation}{section}						% Number equations within sections (i.e. 1.1, 1.2, 2.1, 2.2 instead of 1, 2, 3, 4)
%\numberwithin{figure}{section} 						% Number figures within sections (i.e. 1.1, 1.2, 2.1, 2.2 instead of 1, 2)
\numberwithin{table}{section} 							% Number tables within sections (i.e. 1.1, 1.2, 2.1, 2.2 instead of 1, 2, 3, 4)
\setlength\parindent{0pt} 								% Removes all indentation from paragraphs
\usepackage{listings}									% http://ctan.org/pkg/listings
\renewcommand{\lstlistingname}{Código}	
%\newcommand{\horrule}[1]{\rule{\linewidth}{#1}} 		% Create horizontal rule command with 1 argument of height
\usepackage{url}									% http://ctan.org/pkg/listings
\title{Análisis y diseño de una base de datos para un portal de adopción de mascotas} 			% Title
%\horrule{0.5pt} \\[0.4cm] 								% Thin top horizontal rule	Title rule
%\huge Assignment Title \\ 								% The assignment title
%\horrule{2pt} \\[0.5cm] 								% Thick bottom horizontal rule
\author{												% Authors begin.
Héctor F. \textsc{Jiménez S.}\\
\texttt{hfjimenez@utp.edu.co} \\
\texttt{PGP KEY ID: 0xB05AD7B8}
} 												       % End of  Author name
\date{}    						                       % Date for the report, this will hide the \today.
\begin{document}
\maketitle                      			           % Insert the title, author and date
\begin{center}
\begin{tabular}{l r}								   % two column to
Fecha de Entrega: & \textbf{17 de Febrero de 2017} . \\				   	
Profesor: & Ing.Carlos Alberto Ocampo Sepulveda
\end{tabular}
\end{center}
\section{Definición del problema}
\textbf{www.Adoptame.us} es una iniciativa que pone de manifiesto el compromiso y esfuerzo en tiempos modernos del sector público y empresas del sector privado con el fin de ayudar en el progreso social, enfrentando la problemática actual que viven las mascotas  y animales callejeros de la ciudad de Pereira para disminuir la cantidad de agresiones presentadas por estos a transeúntes; como dato importante el 73\% de las agresiones de animales en todo el sector de Risaralda son producidas por perros callejeros y de los \emph{2.244} casos de agresiones de animales que se registraron durante el 2012 en Risaralda se relacionan con los perros callejeros, de los cuales se estima alrededor de \emph{8000} perros en la ciudad de Pereira, sin contar con otro tipo de animales. El sitio web \textbf{www.Adoptame.us} sera un medio que ayudará a mitigar la situación , permitiéndolo a los ciudadanos reportar casos de abuso animal de forma anónima, realizar adopciones, reporte de animales callejeros, y donaciones al proyecto social. 
\section{Requisitos}
El portal web \textbf{www.Adoptame.us} para poder llevar acabo un producto minimamente viable debe contar con las siguientes caracteristicas:
\begin{itemize}
  \item Posibilidad de realizar un registro de usuarios. 
  \item Cada usuario debe tener un perfil donde se vea su actividad dentro del sitio.
  \item Debe existir un formulario  para resolver quejas o inquietudes. 
  \item Se debe disponer de un formulario de contacto, que capture emails para distribuir campañas. 
  \item Se debe poder realizar el reporte de abuso de animales de forma anónima. 
  \item Debe existir una forma de almacenar información sobre los animales reportados.
  \item Debe existir una forma de realizar donaciones de manera segura y eficiente con una pasarela de pagos.
  \item Dentro del portal debe existir un rol de administrador que pueda acceder a  todo tipo de información. 
  \item El rol de administrador debe poder obtener reportes de cantidad de usuarios registrados,mascotas, comentarios. 
\end{itemize}

Los usuarios de nuestro portal web seguirán realizaran el siguiente procedimiento para realizar \textit{\textbf{adopción}}:
\begin{enumerate}
\item Las personas observan en el sitio web los animales disponibles para adoptar
\item Una vez identificado el animalito deseado el usuario se registra en la plataforma. 
\item A través del perfil creado se realiza una solicitud.
\item Las animalistas y cuidadoras realizan vía telefónica una entrevista con la persona. 
\item Las animalistas envían confirmación por el sistema a la persona con el lugar, y 
fecha de recogida.
\end{enumerate}

Para que los usuarios realicen \textit{\textbf{reporte de animales callejeros}}:
\begin{enumerate}
\item Las personas deben ser usuarios, por lo que deben registrarse en la plataforma. 
\item A través del perfil creado se realiza una solicitud.
\item Las animalistas y cuidadoras validan esto asistiendo al lugar, para llevar el animal a un refugio. 
\item Los animalistas registran la mascota en el portal web, con vacunas. 
\end{enumerate}


Para realizar \textit{\textbf{reporte de Abusos de forma anónima}}:
\begin{enumerate}
\item Las personas usan una url que enmascara la dirección ip original. 
\item El denunciante completa los campos requeridos, como ubicación del abuso, fechas y horas y que continuidad tienen.
\item la denuncia se analiza de manera manual por un animalista para filtrar los detalles.
\item se tramita la denuncia usando la API de la Policía de protección animal(\textit{OPCIONAL!!!})
\end{enumerate}

Para realizar \textit{\textbf{Donaciones}}:
\begin{enumerate}
\item Las personas visitan un link que abrira un link para llenar la información necesaria por Paypal o Western Union. 
\end{enumerate}

Para realizar envio de mensajes, el usuario  esta :
\begin{enumerate}
\item Se llena el formulario de contacto. 
\end{enumerate}


\section{Selección de Tecnologías}
Para realizar este proyecto utilizaremos algunas de las tecnologías que son tendencias en el mundo del desarrollo web, ademas utilizare un sistema gestor de base de datos robusto y compatible con el framework.
\subsection{Sistema Gestor de Base de Datos}
Para desarrollar el proyecto se ha decidió utilizar PostgreSQL corriendo en un entorno Gnu\-Linux por las siguientes razones :
\begin{enumerate}
\item \emph{ Mayor compatibilidad con nuestro web framework.}
\item Implementa el 90\% del estandar SQL
\item Postgres intenta ser un sistema de bases de datos de muy alto nivel, a la altura de Oracle, Sybase o Interbase.
\item Fácil interacción con las bases de datos a través de su cliente \emph{pgsql}.
\item Soporta todo tipo de optimización en el desempeño\footnote{Revisar el libro :SQL Performance Explained}
\item Facilidad para realizar replicas y pruebas de integridad. 
\item La documentación es mejor explicada que en otras bases de datos como MySql
\end{enumerate}
\fnurl{Revisar}{https://medium.com/holistics-software/why-you-should-use-postgres-over-mysql-for-analytics-purpose-e3df42df35d7}.
  
\subsection{Tecnologías Web}
Para desarrollar el proyecto se utilizaran las siguientes tecnologías :
\begin{itemize}
  \item Django Web Framework
  \item CSS3.
  \item HTML5.
  \item Bootstrap v3.3.
  \item Twitter API. 
\end{itemize}
\section{Agradecimientos}
Al profesor Carlos Alberto por las explicaciones pertinentes y los buenos aportes hechos a esta propuesta de proyecto futura.
\section{Webgrafia}
\begin{enumerate}
	\item Envenamiento de Perros callejeros, \url{http://www.elespectador.com/noticias/nacional/envenenan-16-perros-callejeros-pereira-articulo-575320}
	\item Albergue para animales callejeros, \url{http://www.eldiario.com.co/seccion/LOCAL/nuevo-albergue-para-perros-callejeros111118.html}
	\item Estadisticas \url{http://eldiario.com.co/seccion/LOCAL/pereira-tiene-52-mil-perros-1306.html}
\end{enumerate}

\end{document}
